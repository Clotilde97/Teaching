\documentclass[a4paper,10pt]{article}
\usepackage{amsmath}
\usepackage[english]{babel}
\usepackage{graphicx}
\usepackage{enumerate}
\usepackage{hyperref}
\usepackage{float} 
\usepackage{amsfonts}
\parskip 0.1in
 \usepackage[english]{babel}
\usepackage[utf8]{inputenc}
\usepackage{fancyhdr}
 
\pagestyle{fancy}
\fancyhf{}
\rhead{G28 Workshop Three}
\lhead{Charles Rahal}
\rfoot{Page \thepage}
 
\begin{document}
\small
\subsubsection*{Econometrics with Financial Applications: Workshop Three}
\begin{enumerate}
\item Download the example workfile from Lutkepol (2007) from Canvas or from:
\begin{center}
\url{http://www.eviews.com/EViews8/data/wgmacro.wf1}
\end{center}
\item Import the workfile automatically using the new command:
\begin{center}
\texttt{import $<$yourfilepath$>\backslash$wgmacro}
\end{center}
Adjust the sample range to include all observations with command: \texttt{smpl @all}.
\item Plot the three series in levels, log levels and log first differences using the three pre-generated series in the workfile:
\begin{center}
\texttt{graph tsplot1.line consumption income investment\\
graph tsplot2.line lconsumption lincome linvestment\\
graph tsplot3.line dlconsumption dlincome dlinvestment\\}
\end{center}
\item Using the plots, unit root tests (Workshop Two) (\texttt{\{\%Series\}.uroot}) and autocorrelation functions (\texttt{\{\%Series\}.correl}), assess the stationarity of each series.
\item Assuming the three series are I(1), estimate a VAR(4) model in log first differences with least squares in the following three ways: 1.) \emph{Select all three variables}$\rightarrow$\emph{Open}$\rightarrow$\emph{As VAR}, 2.) \emph{Object}$\rightarrow$\emph{New Object}$\rightarrow$\emph{VAR} and type in your three endogenous variables (dlincome, dlinvestment, dlconsumption) with lag lengths 1 4 (\emph{why?}), 3.) In the command window:
\begin{center}
\texttt{var var1.ls 1 4 dlincome dlinvestment dlconsumption}
\end{center}
\item Lets now consider the number of lags to use: \emph{View}$\rightarrow$\emph{Lag Structure}$\rightarrow$\emph{Lag Length Criteria} and then re-estimate the model using the preferred lag length:
\begin{center}
\texttt{var1.laglen(8)\\
var var1.ls 1 2 dlincome dlinvestment dlconsumption}
\end{center}
See the accompanying program for how to do write a script to determine and estimate a var model with a suitable lag length.
\item Lets now test for block-exogenity and Granger Causality. Using the inbuilt EViews menus:\emph{View}$\rightarrow$\emph{Lag Structure}$\rightarrow$\emph{Granger Causality/Block Exogeneity}. Alternatively, you can use:
\begin{center}
\texttt{var1.testexog}
\end{center}
How do we interpret these results? What can we conclude?
\item Lets check our results by doing individual Wald tests. This is more tricky. We have to make the VAR into a system then estimate it by OLS: \emph{Proc}$\rightarrow$\emph{Make System}$\rightarrow$\emph{Order by Variable}$\rightarrow$\emph{Proc}$\rightarrow$\emph{Estimate}$\rightarrow$\emph{View}$\rightarrow$\emph{Coefficient Diagnostics}$\rightarrow$\emph{Wald Coefficient Tests}. Test out some restrictions which correspond to the block-exogenity and Granger Causality tests which you undertook on the VAR in menus - can you get the same numbers? For command line/programming guidance on this - see the program on the following page.
\item Finally, estimate the impulse responses to each variable from a shock from each other variable by using the menus: \emph{View}$\rightarrow$\emph{Impulse Response}, or by using the command line:
\begin{center}
\texttt{var1.impulse(10,m,a,se=a) dlconsumption dlincome dlinvestment @ dlconsumption dlincome dlinvestment}
\end{center}
How do we interpret the results form this?\\
\item \textbf{Optional Homework:} Change the impulse defintions and the orderings to see how the effects change.
\end{enumerate}
\newpage
\subsubsection*{Program for Workshop Three}
import c:$\backslash$wgmacro\\
!maxvarlags=4\\
matrix(3,!maxvarlags) ic\_mat\\
graph tsplot1.line consumption income investment\\
graph tsplot2.line lconsumption lincome linvestment\\
graph tsplot3.line dlconsumption dlincome dlinvestment\\
rename dlconsumption dlcons\\
rename dlincome dlinc\\
rename dlinvestment dlinv 'rename due to EViews 24 character limit\\
for \%series dlcons dlinc dlinv\\	
\indent	for \%test adf kpss\\
\indent \indent		for \%det const trend\\
\indent \indent \indent		freeze(\{\%series\}\_\{\%det\}\_\{\%test\}\_d0)	\{\%series\}.uroot(\{\%test\},\{\%det\},dif=0)\\
\indent \indent \indent		freeze(\{\%series\}\_\{\%det\}\_\{\%test\}\_d1)	\{\%series\}.uroot(\{\%test\},\{\%det\},dif=1)\\
\indent \indent \indent		freeze(\{\%series\}\_\{\%det\}\_\{\%test\}\_d2)	\{\%series\}.uroot(\{\%test\},\{\%det\},dif=2)\\
\indent \indent		next\\
\indent	next\\
\indent	freeze(\{\%series\}\_d0) \{\%series\}.correl\\
\indent	freeze(\{\%series\}\_d1) d(\{\%series\},1).correl\\
\indent	freeze(\{\%series\}\_d2) d(\{\%series\},2).correl\\
next\\
!aic = 999999999999999999999\\
!schwarz=99999999999999999\\
!hq=9999999999999999999999\\
for !p = 1 to !maxvarlags	\\
\indent	var var{!p}.ls 1 !p dlcons dlinc dlinv\\
\indent	!iccounter=1\\
\indent	for \%IC aic schwarz hq\\
\indent \indent		ic\_mat(!iccounter,!p)=var1.@\{\%IC\}\\
\indent \indent		if var\{!p\}.@\{\%IC\}$<$!\{\%IC\} then\\
\indent \indent	\indent	!bestlag\{\%IC\}=!p\\
 \indent \indent \indent			!\{\%IC\}=var\{!p\}.@\{\%IC\}\\
\indent \indent		endif\\
\indent	!iccounter=+!iccounter+1\\
\indent	next	\\
next\\
var varaic.ls 1 !bestlagaic dlcons dlinc dlinv\\
var varschwarz.ls 1 !bestlagschwarz dlcons dlinc dlinv\\
var varhq.ls 1 !bestlaghq dlcons dlinc dlinv\\
varaic.laglen(8)\\
varaic.testexog\\
varaic.makesystem(n=varsystem)\\
varsystem.ls\\
varsystem.wald c(3)=c(4)=0	'granger causality test\\
varsystem.wald c(3)=c(4)=c(5)=c(6)=0 'block exogeneity test\\
varaic.impulse(10,m,a,se=a) dlcons dlinc dlinv @ dlcons dlinc dlinv\\
\end{document}