\documentclass[10pt]{beamer}
\setbeamertemplate{navigation symbols}{}%remove navigation symbols
\usepackage{booktabs}
\usepackage{subfig}
\usepackage[newfloat]{minted}
\usepackage{listings}
\usepackage{lmodern}
\usepackage[T1]{fontenc}
\useoutertheme{infolines}
\usepackage[framemethod=tikz]{mdframed}
\usetikzlibrary{shadows}
\newmdenv[shadow=true,shadowcolor=black,font=\sffamily,rightmargin=8pt]{shadedbox}
\usepackage{pifont,xcolor}% http://ctan.org/pkg/{pifont,xcolor}
\definecolor{myblue}{RGB}{49,54,149}
\definecolor{myred}{RGB}{165,0,38}
\usepackage{graphicx}
\usepackage{caption}

\newcommand{\lenitem}[2][.55\linewidth]{\parbox[t]{#1}{#2\strut \strut}}

\usetheme{Frankfurt}

   \usefonttheme{professionalfonts} 
\setbeamertemplate{itemize items}[circle]

\newenvironment{variableblock}[3]{%
  \setbeamercolor{block body}{#2}
  \setbeamercolor{block title}{#3}
  \begin{block}{#1}}{\end{block}}
\usecolortheme{dove}
\usepackage{fancybox}
\setbeamercolor{block title}{bg=white,fg=black}
\newenvironment{fminipage}%
{\begin{Sbox}\begin{minipage}}%
{\end{minipage}\end{Sbox}\fbox{\TheSbox}}

\newcommand{\itemcolor}[1]{% Update list item colour
  \renewcommand{\makelabel}[1]{\color{#1}\hfil ##1}}

\newcounter{tmpc}
%\usefonttheme{structuresmallcapsserif}
\setbeamercolor{section in head/foot}{fg=black, bg=white}

\setbeamertemplate{frametitle}
{
    \nointerlineskip
    \begin{beamercolorbox}[sep=0.3cm,ht=1.8em,wd=\paperwidth]{frametitle}
        \vbox{}\vskip-3ex%
        \strut\insertframetitle\strut
        \vskip-1.2ex%
    \end{beamercolorbox}
}
\addtobeamertemplate{frametitle}{\vskip0.5ex}{}
\makeatletter
\setbeamertemplate{footline}
{
  \leavevmode%
  \hbox{%
  \begin{beamercolorbox}[wd=.875 \paperwidth,ht=2.25ex,dp=1ex,left]{section in head/foot}%
    \usebeamerfont{author in head/foot}\quad \quad \insertshorttitle
 \end{beamercolorbox}%
 \begin{beamercolorbox}[wd=.125\paperwidth,ht=2.25ex,dp=1ex,right]{section in head/foot}%
    \usebeamerfont{date in head/foot}\insertshortdate \quad \quad
    \insertframenumber{} / \inserttotalframenumber\hspace*{2ex} 
  \end{beamercolorbox}}%
  \vskip0pt%
}
\makeatother

\def\mf{
\begin{itemize}
\item Item
\end{itemize}
}


\section{Title/Intro}
\subsection{Subsection}

\usepackage{listings}

\lstdefinestyle{BashInputStyle}{
  language=bash,
  basicstyle=\small\sffamily,
  numbers=left,
  numberstyle=\tiny,
  numbersep=3pt,
  frame=tb,
  columns=fullflexible,
  backgroundcolor=\color{white},
  linewidth=0.9\linewidth,
  xleftmargin=0.1\linewidth
}


\begin{document}

\title{An Introduction to the Command Line}
\date{}
\author{Charles Rahal and Felix Tropf \\ Department of Sociology, University of Oxford}
\frame{\titlepage 
\begin{center}
\vspace{-0.5in}
\texttt{25th June, 2017}\\ \vspace{0.2in}
\color{myblue}NCRM Summer School - University of Oxford\color{black}\\ \vspace{0.1in}
Slides, code, lecture notes: \color{myred}\url{https://github.com/crahal/Teaching}\color{black}\\ \vspace{0.1in}
Comments/questions/suggestions: \color{myred}\url{charles.rahal@sociology.ox.ac.uk}\\ \vspace{0.1in}
%Please note: if there is anyone here\color{red}\textbf{extremely preliminary}\color{black}.\\ \vspace{0.35in}
\end{center}
}


\subsection{}
\frame{
\frametitle{Some Class Administration}
\begin{itemize}
\item \textbf{WELCOME!} - to the NCRM Summer School in SocioGenomics! \pause \vspace{0.1in}
\item I'd like to also introduce \href{http://felix-tropf.com/}{\texttt{Felix}}: the TA for this class.\pause \vspace{0.1in}
\item This is the first of several computer lab workshops - each will focus on different things: e.g. - I'll TA the class ran by Felix after this one on Plink. \pause \vspace{0.1in}
\item Note: from the pre-sessional materials, you should have set up a command line environment.\vspace{0.05in}
\begin{itemize}
\item \textbf{Windows:}, this will involve a virtual machine. If you haven't managed this, I'll help you directly after this session or this evening before or after dinner.\vspace{0.05in}
\item \textbf{Linux or Mac:} you already have a command line -- nothing extra required. \pause \vspace{0.1in}
\end{itemize}
\item However: For a lot of classes, this isn't necessary, where a simple R (more often) or Python (less often) will be more than sufficient. \pause \vspace{0.1in}
\item The CLI will only be required for some classes, but practicing with it this week provides \textbf{invaluable} experience for HPCs (an integral part of genomics).
\end{itemize}
}


\subsection{}
\frame{
\frametitle{Why Use the Command Line?}
We definitely need to first further motivate \emph{why} we want to use the command line. Although there are many reasons for utilizing it, and conversely many reasons for maintaining a GUI for a lot of tasks, CLIs are important for these reasons especially:\vspace{0.1in}\pause
\begin{itemize}
\item Genetics datasets are \textbf{BIG}. The just released UK Biobank dataset is about 8tb. This necessitates the use of \textbf{H}igh \textbf{P}erformance \textbf{C}omputers (HPCs) -- this is due to the reduced cost of genomic sequencing (specifically NGS).\vspace{0.1in}\pause
\item Automation/Replication:  No matter how complex the command or series of commands, the CLI doubles as a scripting language (shell scripts - which we will briefly discuss if there's time) which can be ran as a batch.\vspace{0.1in}\pause
\item Control: Commands are often more powerful and precise, giving more control over the operations to be performed\vspace{0.1in}\pause
\item Software tools: A number of genetics software tools will only run in Linux/Mac environments.  
\end{itemize}
}


\subsection{}
\frame{
\frametitle{Our Friend - ARCUS (or AWS)}
\begin{itemize}
\item Why have this class at all? Our friend ARCUS \href{http://www.arc.ox.ac.uk/content/introduction-linux}{\color{blue}provides some motivation}:\vspace{0.1in}
\begin{center}
   \includegraphics[width=0.75\textwidth]{figure1.png}
\end{center}
\pause
\item The CLI can seem like a harsh, unforgiving environment.\vspace{0.1in} \pause
\item \href{http://www.urbandictionary.com/define.php?term=rm\%20-rf\%20\ q\%2F}{\color{blue}Urban Dictionary} calls the famous CLI command \texttt{rm -rf}:
\begin{center}
 \emph{`the finest compression avaliable under UNIX/Linux...\\ unfortunatly, there is no decompressor avaliable.`}\vspace{0.1in}\pause
 \end{center}
\item However: \textbf{the potential is endless!}\vspace{0.1in}\pause
\item (and a Virtual Machine is the best way to learn).
\end{itemize}
}

\subsection{}
\frame{
\frametitle{This isn't a scary course!}
Melinda told me that a good strategy is to put a picture of your cat on the slides to persuade participants that the course isn't scary...  \pause
\begin{figure}[!b]
\centering  
\caption{This isn't a scary course!}\label{figure_5}
\subfloat[Sleepy]{\label{figure_5a}\includegraphics[width=0.35\textwidth]{moonpie1.png}}\hspace{0.4in}
\subfloat[Curious]{\label{figure_5b}\includegraphics[width=0.34\textwidth]{moonpie3.png}}\\
\normalsize
\end{figure}
%And now is probably the best time to do that.
}


\subsection{}
\frame{
\frametitle{A Brief History of nix*-like environments}
\begin{itemize}
\item Before we discuss \emph{exactly} what the CLI is, we should briefly talk about the history of what is now known as *nix-like computing.\vspace{0.1in}\pause
\item Unix was created 1969 at Bell Laboratories -- then a development division at AT\&T -- by Ken Thompson and Dennis Richie.\vspace{0.1in}\pause
\item All software previously hardware dependent, written in assembler language.\vspace{0.1in}\pause
\item However -a critical feature of Unix was that it is proprietary.\vspace{0.1in}\pause
\item U. of California, Berkley developed BSD -- based on the AT\&T code base. Commercial Unix derivatives emerged through the 1980s.\vspace{0.1in}\pause
\item However, the breakthrough came in September 1991, when Linus Torvalds released the Linux operating system kernel.\vspace{0.1in}\pause
\item \textbf{Utilized everywhere:} i.e. Android operating system and Chrome OS which runs on Chromebooks (televisions,  smart-watches, servers, etc).\vspace{0.1in}\pause
\end{itemize}
}

\subsection{}
\frame{
\frametitle{The File Tree}
\vspace{-.3in}
\begin{center}
   \includegraphics[width=0.75\textwidth]{Infographic2.png}
\end{center}
You might also find  \texttt{/opt:} optional software or \texttt{/tmp:} temporary space.
}

\subsection{}
\frame{
\frametitle{System Architecture}
\begin{center}
   \includegraphics[width=0.75\textwidth]{Infographic1.png}
\end{center}
Importantly, the shell is a program that takes your commands from the keyboard and gives them to the operating system to perform.
}

\subsection{}
\frame{
\frametitle{The Terminal}
The terminal is a program which lets you interact with the shell:\vspace{0.1in}
\begin{center}
   \includegraphics[width=0.75\textwidth]{theterminal.png}
\end{center}
If the last character of your shell prompt is \# rather than \$, you are operating as the superuser. This means that you have administrative privileges. This can be potentially dangerous, since you are able to delete or overwrite any file on the system. Unless you absolutely need administrative privileges, do not operate as the superuser (try \texttt{whoami} and \texttt{sudo whoami}).
}

\section{Fundamental CLI Commands}
\subsection{}
\begin{frame}[fragile]{Examples}
\frametitle{Fundamental CLI Commands: \texttt{man}}
\begin{itemize}
\item The \texttt{man} command shows the manual for a given page, including information on options and usage.\vspace{0.1in}
\item This is the default resource for getting help on specific commands.\vspace{0.1in}
\begin{lstlisting}[style=BashInputStyle]
user@system:~$ man echo
ECHO(1)                  User Commands                 ECHO(1)

NAME
       echo - display a line of text

SYNOPSIS
...
\end{lstlisting}
\item For example, the \texttt{man echo} command will display the manual for the \texttt{echo} command which we just saw.\\
\end{itemize}
\end{frame}


\subsection{}
\begin{frame}[fragile]{Examples}
\frametitle{Fundamental CLI Commands: \texttt{whatis}}
\begin{itemize}
\item The whatis command gives a short summary description of the specific command, and command inputs can be stacked together. \vspace{0.2in}
\begin{lstlisting}[style=BashInputStyle]
user@system:~$ whatis echo ls
echo (1)             - display a line of text
ls (1)               - list directory contents
\end{lstlisting}
\vspace{0.1in}
\item Each manual page has a short description available within it -- and \texttt{whatis} searches the manual page names.
\end{itemize}
\end{frame}




\subsection{}
\begin{frame}[fragile]{Examples}
\frametitle{Fundamental CLI Commands: \texttt{ls}}
\begin{itemize}
\item \texttt{ls} is a Linux shell command that lists directory contents of files and directories.\vspace{0.2in}
\end{itemize}
\begin{lstlisting}[style=BashInputStyle]
user@system:~$ ls
Desktop Documents Downloads Music Pictures Public Videos
\end{lstlisting}
\vspace{0.1in}
\begin{itemize}
\item The lists can also be sorted, accept wildcards, and pipe outputs to file. \vspace{0.1in}
\item It can show hidden files, show file size, and has the option for a `long' format.\vspace{0.1in}
\end{itemize}
\end{frame}

\subsection{}
\begin{frame}[fragile]{Examples}
\frametitle{Fundamental CLI Commands: \texttt{pwd}}
\begin{itemize}
\item However, this isn't much use without knowing \emph{where} we are in the file tree. \vspace{0.2in}
\end{itemize}
\begin{lstlisting}[style=BashInputStyle]
user@system:~$ pwd
/home/user
\end{lstlisting}
\vspace{0.1in}
\begin{itemize}
\item In *nix-like operating systems, the \texttt{pwd} command (which stands for print working directory) writes the full path-name of the current working directory to the standard output.\\
\end{itemize}
\end{frame}

\subsection{}
\begin{frame}[fragile]{Examples}
\frametitle{Fundamental CLI Commands: \texttt{mkdir} and \texttt{cd}}
\begin{itemize}
\item We should make a new directory (or folder -- a container for other files) for the purposes of following along with these examples. \vspace{0.2in}
\end{itemize}
\begin{lstlisting}[style=BashInputStyle]
user@system:~$ mkdir nix
user@system:~$ ls
nixTextbookChapter
user@system:~$ cd nix
user@system:~/nix$ pwd
/home/user/nix
\end{lstlisting}
\vspace{0.1in}
\begin{itemize} 
\item This brings us to \emph{absolute} and \emph{relative} paths. \vspace{0.1in}
\item From the documents folder above, we can navigate to the new folder (\texttt{nix}) through the relative path or through the absolute path. \vspace{0.1in}
\item The \texttt{cd} command is used to change the current directory.\vspace{0.1in}
\item There are multiple ways of identifying the same file path in the directory tree.\vspace{0.1in}
\end{itemize}
\end{frame}

\subsection{}
\begin{frame}[fragile]{Examples}
\frametitle{Fundamental CLI Commands: \texttt{touch}}
\begin{itemize}
\item The \texttt{touch} command is the easiest way to create new, empty files...
\end{itemize}
\begin{lstlisting}[style=BashInputStyle]
user@system:~/nix$ touch testfile
user@system:~/nix$ ls
testfile
\end{lstlisting}
\vspace{0.1in}
\begin{itemize}
\item \texttt{testfile} might seem unfamiliar to you for one obvious reason: the lack of an extension. \vspace{0.1in}
\item  In *nix-like systems, the extension is ignored, and the file type is determined automatically (and the command \texttt{file} can tell us additional information).\vspace{0.1in}
\end{itemize}
\end{frame}

\subsection{}
\begin{frame}[fragile]{Examples}
\frametitle{Fundamental CLI Commands: \texttt{rm}}
\begin{itemize}
\item Now that we've learnt how to create files and directories, lets look at removing them:
\end{itemize}
\begin{lstlisting}[style=BashInputStyle]
user@system:~/nix$ cd ..
user@system:~/$ rm -ri nix
rm: remove directory 'nix'?
\end{lstlisting}
\vspace{0.1in}
\begin{itemize}
\item This shows how we can stack options together, where \texttt{rm -ri} is the equivalent to \texttt{rm -r -i}. \vspace{0.1in}
\item  However, command line options are not universal between commands, and implementation across different operating systems may vary.\vspace{0.1in}
\item We should also note the existence of \texttt{rmdir}, the negative equivalent to \texttt{mkdir}, which removes \emph{empty} directories.
\end{itemize}
\end{frame}

\subsection{}
\begin{frame}[fragile]{Examples}
\frametitle{Fundamental CLI Commands: \texttt{mv} and \texttt{cp}}
\begin{itemize}
\item It seems only natural that we now learn how to move, copy and rename files.
\end{itemize}
\begin{lstlisting}[style=BashInputStyle]
user@system:~/nix$ cp testfile ..
user@system:~/nix$ rm ../testfile
user@system:~/nix$ mv testfile ..
\end{lstlisting}
\vspace{0.1in}
\begin{itemize}
\item We need not actually remove \texttt{testfile} after copying it and before moving it, as the \texttt{mv} command would simply overwrite it.\vspace{0.1in}
\item To rename a file, we can just move it with a new name. 
\end{itemize}
\end{frame}

\subsection{}
\begin{frame}[fragile]{Examples}
\frametitle{Other Basic Utilities}
\begin{itemize}
\item We can \texttt{clear} the terminal screen.\vspace{0.1in}
\item We can display all environmental variables with \texttt{env} (try \texttt{echo \$LANG}).\vspace{0.1in}
\item \texttt{find} and \texttt{locate} search for files and directories (with the latter being faster -- performing on a database of indexed filenames).\vspace{0.1in}
\item  \texttt{date} displays or sets (with the option \texttt{-s}) the system time, and \texttt{cal} displays the calender.\vspace{0.1in}
\item \texttt{history} and specifically - \texttt{history [NUM]} reports the last \texttt{[NUM]} commands.\vspace{0.1in}
\item We can \texttt{exit} from the current terminal session, \texttt{logout} as a specific user, or \texttt{shutdown} the machine entirely.
\end{itemize}
\end{frame}

\section{Slightly More Advanced Concepts}
\subsection{}
\begin{frame}[fragile]{Examples}
\frametitle{Editors at the Command Line}
\begin{itemize}
\item Lets introduce the use of text editors at the command line (vi, pico, nano and emacs). These are plain text editors -- not word processing suites.\vspace{0.1in}
\item Unlike many GUI based editors, the mouse does not move the cursor. Unlike PC editors, you cannot modify text by highlighting it with a mouse.\vspace{0.1in}
\item Lets use \texttt{nano} to create a file which we will use for some examples later on: a list of all the fruits we can think of (call the file \texttt{allfruits}). \vspace{0.1in}
\end{itemize}
\begin{table}
\caption*{Introducing Text Editors at the CLI: \texttt{nano allfruits}}
\begin{lstlisting}[style=BashInputStyle]
Apple
Apricot
Avocado
.
.
"allfruits" 90 lines, 846 characters
\end{lstlisting}
\end{table}
\end{frame}







\subsection{}
\begin{frame}[fragile]{Examples}
\frametitle{I/O Redirection}
\begin{itemize}
\item Most command line programs output their results to the ‘standard output’ (STDOUT): which, by default, is the display.
\item We can redirect standard output to specific files using the ‘>’ character, and append to a file using ‘> >’.
\item Commands can also accept input from ‘standard input’ (STDIN), which defaults to the keyboard: to redirect from STDIN, we use the ‘<’ character.
\end{itemize}
\begin{table}
%\caption*{Introducing Text Editors at the CLI: \texttt{nano allfruits}}
\begin{lstlisting}[style=BashInputStyle]
user@system:~/nix$ ls > filelist.txt
user@system:~/nix$ sort < filelist.txt
\end{lstlisting}
\end{table}
\begin{itemize}
\item There is also a third stream which will go unexamined (‘standard error’ or STDERR) which is used for error messages – and also defaults to the terminal.
\end{itemize}
\end{frame}

\subsection{}
\begin{frame}[fragile]{Examples}
\frametitle{Filters}
\emph{filters} are a class of programs which are extremely useful for I/O Redirection:\vspace{0.1in}
\begin{itemize} 
\item \texttt{sort:} Sorts STDIN and outputs the sorted result on standard output.\vspace{0.05in}
\item \texttt{uniq}: Given a sorted stream of data from STDIN, it removes duplicates.\vspace{0.05in}
\item \texttt{grep}: Examines each line of data it receives from standard input and outputs every line that contains a specified pattern of characters.\vspace{0.05in}
\item \texttt{head}: Outputs the first few lines of its input (defaults to first 10 lines).\vspace{0.05in}
\item \texttt{cat}: Concatenates files and displays their contents, or just displays contents if given one input.\vspace{0.05in}
\item \texttt{cut}: Divides a file into several columns.\vspace{0.05in}
\item \texttt{nl}: prints the line number before data.\vspace{0.05in}
\item \texttt{wc}: which prints the count of lines, words and characters.\vspace{0.05in}
\end{itemize}
\end{frame}

\subsection{}
\begin{frame}[fragile]{Examples}
\frametitle{The Pipe}
\begin{itemize}
\item The pipe (\texttt{|}), which allows you to connect multiple commands together.\vspace{0.1in}
\item The standard output of one command is fed into the standard input of another, enabling you to chain together individual commands to create something really powerful.\vspace{0.1in}
\item The example below \emph{pipes} the output from \texttt{ls} into the \texttt{head} command which takes the input \texttt{1} to show us the first 1 line of the \texttt{ls} output: we've chained the two commands together.
\end{itemize}
\begin{table}
%\caption*{Introducing Text Editors at the CLI: \texttt{nano allfruits}}
\begin{lstlisting}[style=BashInputStyle]
user@system:~/nix$ ls | head -1
allfruits
\end{lstlisting}
\end{table}
\end{frame}

\subsection{}
\begin{frame}[fragile]{Examples}
\frametitle{Wildcards}
\begin{itemize}
\item Another more advanced concept is the `wildcard': a set of tools that allow you to create a pattern which defines a specific set of files or directories.\vspace{0.1in}
\item There are two simple types of wildcards: 
\setbeamertemplate{enumerate items}[default]
\begin{enumerate}
\item \texttt{*} represents zero or more characters
\item \texttt{?} represents a single character.
\end{enumerate}
\begin{table}
%\caption*{Introducing Text Editors at the CLI: \texttt{nano allfruits}}
\begin{lstlisting}[style=BashInputStyle]
user@system:~/nix$ touch fileonea fileoneb filetwoa filetwob
user@system:~/nix$ ls file*a
fileonea filetwoa
user@system:~/nix$ ls filetwo?
filetwoa filetwob
\end{lstlisting}
\end{table}
\item A regular expression (or `regex') includes such functionality, but is a much more powerful pattern matcher beyond the scope of this introduction.
\end{itemize}
\end{frame}

\subsection{}
\begin{frame}[fragile]{Examples}
\frametitle{Users, Groups and Permissions}
\begin{itemize}
\item Permissions  specify what a user can and cannot do.\vspace{0.05in}
\item  Perhaps, for example, you want to lock your files so other people cannot change them or secure system files from damage. \vspace{0.05in}
\item Permissions are split into three distinct categories which govern the ability to: Read: \texttt{r}, Write: \texttt{w} and Execute: \texttt{x}.\vspace{0.05in}
\item For every file, we need to define permissions for potential users: 
\begin{itemize}
\item The user who created the file (u).
\item The group which owns the file (g).
\item Others (o).\vspace{0.05in}
\end{itemize}
\item There are typically only two people who can manage the permissions of a given file or directory: the owner and the root user. \vspace{0.05in}
\item To view the permissions associated with an individual file, we can use the \texttt{-l} option on the \texttt{ls} command:\\
\begin{lstlisting}[style=BashInputStyle]
user@system:~/nix$ ls -l allfruits
-rw-rw-r-- 1 user user 846 May 29 14:57 allfruits
\end{lstlisting}
\end{itemize}
\end{frame}

\subsection{}
\begin{frame}[fragile]{Examples}
\frametitle{Users, Groups and Permissions (Cont.)}
\begin{itemize}
\item The first character determines whether it is a file (\texttt{-}) or a directory (\texttt{d}).\vspace{0.05in}
\item We have information on the permissions for \texttt{u}, \texttt{g} and \texttt{o}.\vspace{0.05in}
\item A \texttt{-} represents the omission of a permission.\vspace{0.05in}
\item To change the permissions, use the \texttt{chmod}, including information on: 1.) Who are we changing the permission for? 2.) Are we giving (\texttt{+}) or taking permission (\texttt{-})?, 3.) Which permissions are we setting? \vspace{0.05in}
\item For example, lets take away read permissions from the group and others:
\begin{lstlisting}[style=BashInputStyle]
user@system:~/nix$ chmod og-r allfruits
user@system:~/nix$ ls -l allfruits
-rw------- 1 user user 846 May 29 15:09 allfruits
\end{lstlisting}\vspace{0.05in}
\item Importantly,  permissions are not inherited from the parent directory - and the recursive option (\texttt{-R}) for \texttt{chmod} can be especially useful.\vspace{0.05in}
\item There are a range of `short-hand' commands (e.g. \texttt{chmod 751 <filename>}).
\end{itemize}
\end{frame}

\subsection{}
\begin{frame}[fragile]{Examples}
\frametitle{Archiving}
\begin{itemize}
\item Archiving in our context is similar to the familiar .zip extension in Windows. \vspace{0.05in}
\item Introducing the tape archive -- \texttt{tar}: extract with \texttt{-x}, create with \texttt{-c.}.\vspace{0.05in}
\item Typically compressed with \texttt{gzip} (Lempel-Ziv) or \texttt{bzip2} (Burrows-Wheeler).\vspace{0.05in}
\end{itemize}
\begin{lstlisting}[style=BashInputStyle]
user@system:~/nix$ head -5 allfruits > top5fruits
user@system:~/nix$ ls -la
-rw------- 1 user user  846 May 30 13:32 allfruits
-rw-rw-r-- 1 user user   38 May 30 13:33 first5fruits
user@system:~/nix$ tar -cvf fruits.tar allfruits first5fruits
user@system:~/nix$ gzip fruits.tar
user@system:~/nix$ tar -cvf fruits.tar allfruits first5fruits
user@system:~/nix$ bzip2 fruits.tar
user@system:~nix$ bzip2 ls -l
-rw------- 1 user user 10240 May 30 14:07 allfruits
-rw-rw-r-- 1 user user    38 May 30 13:33 first5fruits
-rw-rw-r-- 1 user user   214 May 30 14:13 fruitlists.tar.bz2
-rw-rw-r-- 1 user user   224 May 30 14:13 fruitlists.tar.gz
\end{lstlisting}\vspace{0.05in}
\end{frame}

\section{Bash Scripting}
\subsection{}
\begin{frame}[fragile]{Examples}
\frametitle{Bash Scripting: Introduction}
\begin{itemize}
\item Bash scripting performs complex, repetitive tasks with minimal effort.\vspace{0.1in}
\item A script is just a text file containing commands which could be ran directly.\vspace{0.1in}
\item It is a convention (albeit unnecessary) to give bash scripts an extension of .sh.\vspace{0.1in}
\item Lets create our first script (\texttt{myfirstscript.sh}) and describe how it works:\vspace{0.1in}
\end{itemize}
\begin{lstlisting}[style=BashInputStyle]
#!/bin/bash
echo Hello There! Welcome to Bash Scripting!
\end{lstlisting}\vspace{0.1in}
\begin{itemize}
\item The first line is called the `shebang'. We can run the file in two ways:\vspace{0.1in}
\end{itemize}
\begin{lstlisting}[style=BashInputStyle]
user@system:~/nix$ ./myfirstscript.sh
Hello There! Welcome to Bash Scripting!
user@system:~/nix$ bash myfirstscript.sh
Hello There! Welcome to Bash Scripting!
\end{lstlisting}\vspace{0.1in}
\end{frame}



\subsection{}
\begin{frame}[fragile]{Examples}
\frametitle{Bash Scripting: Variables}
\begin{itemize}
\item Just like in other languages: variables are temporary methods of storing information.\vspace{0.1in}
\item Setting them requires no \$, but reading them does.\vspace{0.1in}
\item Lets pass variables to a script which accepts \$1 and \$2 as inputs:\vspace{0.1in}
\end{itemize}
\begin{lstlisting}[style=BashInputStyle]
#!/bin/bash
echo Hello $1! Message sent from $2!
\end{lstlisting}\vspace{0.1in}
\quad and then execute it as before:\vspace{0.1in}
\begin{lstlisting}[style=BashInputStyle]
user@system:~/nix$ ./variables.sh Felix Charlie
Hello Felix! Message sent from Charlie!
\end{lstlisting}\vspace{0.1in}
\begin{itemize}
\item There are also a number of special variables: \texttt{\$0}, \texttt{\$n}, \texttt{\$?} etc.
\end{itemize}
\end{frame}

\subsection{}
\begin{frame}[fragile]{Examples}
\frametitle{Bash Scripting: User Input}
\begin{itemize}
\item We can also ask the user for input using \texttt{read}:
\end{itemize}
\begin{lstlisting}[style=BashInputStyle]
#!/bin/bash
echo Hey, Buddy! What is your favorite color?
read favoritecolor
echo Wow! $favoritecolor is my favorite color too!
\end{lstlisting}\vspace{0.1in}
\quad and then execute it as before:\vspace{0.1in}
\begin{lstlisting}[style=BashInputStyle]
user@system:~/nix$ ./readvariable.sh
Hey, Buddy! What is your favorite color?
Blue
Wow! Blue is my favorite color too!
\end{lstlisting}\vspace{0.1in}
\end{frame}

\subsection{}
\begin{frame}[fragile]{Examples}
\frametitle{Bash Scripting: Arithmethic}
\begin{itemize}
\item We can also do simple arithmetic operations using \texttt{let} and \texttt{expr}: \texttt{let} stores the variable, but \texttt{expr} prints it out.\vspace{0.05in}
\item We can utilize the standard operators of \textbf{+}, \textbf{-}, \textbf{$\backslash$*},\textbf{ /}.\vspace{0.05in}
\end{itemize}
\begin{lstlisting}[style=BashInputStyle]
#!/bin/bash
let ''a = $1 + $2''
echo adding $1 and $2 together gives $a
echo multiplying $1 and $2 together gives $(expr $1 \* $2)
\end{lstlisting}\vspace{0.1in}
\quad and then execute it as before:\vspace{0.1in}
\begin{lstlisting}[style=BashInputStyle]
user@system:~/nix$ ./arithmetic.sh 10 5
adding 10 and 5 together gives 15
multiplying 10 and 5 together gives 50
\end{lstlisting}\vspace{0.05in}
\begin{itemize}
\item Finally, we can also embed \texttt{if} statements just like in other languages (no indentation), and utilize various types of loops (while, until and for).\vspace{0.05in}
\end{itemize}
\end{frame}

\section{Python and R}
\subsection{}
\begin{frame}[fragile]{Examples}
\frametitle{Python and R}
\begin{itemize}
\item The lack of a GUI on HPCs requires us to submit our Python and R scripts through the command line rather than an IDE.\vspace{0.05in}
\item This is as simple as writing your R or Python scripts locally and then transferring them (\texttt{scp} or \texttt{rsync}) or writing them using a CLI editor.\vspace{0.05in}
\item Lets create an example file called \texttt{helloworld} which simply has the single line: \texttt{print(''Hello World! How are you?'')}.\vspace{0.05in}
\item Assuming Py 2, we can execute \texttt{helloworld} with almost identical output:\vspace{0.05in}
\begin{lstlisting}[style=BashInputStyle]
user@system:~/nix$ python helloworld
Hello World! How are you?
user@system:~/nix$ Rscript helloworld
[1] ''Hello World! How are you?''
\end{lstlisting}\vspace{0.1in}
\item We can execute the files with \texttt{./} as above, but this requires a \texttt{shebang}! \#!/usr/bin/Rscript for R and \#!/usr/bin/env python. 
\end{itemize}
\end{frame}


\section{Conclusion}
\subsection{}
\begin{frame}[fragile]{Examples}
\frametitle{Conclusion}
\begin{itemize}
\item We've gone through the main basic ideas behind the CLI, with the intention of providing some intuition.\vspace{0.05in}\pause
\item Other courses use these ideas in their application to sociogenomics.\vspace{0.05in}\pause
\item We didn't really cover: alias-ing, networking, \textbf{ssh-ing}, scheduling, awk and sed. We didn't really consider control statements, or regex.\vspace{0.05in} \pause
\item Hopefully you now feel as suave with the CLI as this guy:
\end{itemize}
\begin{figure}[!b]
\includegraphics[width=0.4\linewidth]{clevercat.jpg}
\end{figure}
\end{frame}

\end{document}